\documentclass[10pt,onecolumn]{article}
\usepackage[utf8]{inputenc}
 
\pagenumbering{arabic}
\usepackage{siunitx}
\usepackage{graphicx}
\usepackage{float}
\usepackage{placeins}
\usepackage{adjustbox}
\usepackage{multirow}
\usepackage[ruled,vlined,linesnumbered]{algorithm2e}
\usepackage{tablefootnote}
\usepackage{mathtools}
\usepackage[margin=0.75in]{geometry} %This is for all the margins

\def\@maketitle{%
  \null
  \vskip 2em%
  \begin{center}%
  \let \footnote \thanks
    {\LARGE \@title \par}%
    \vskip 1.5em%
    {\large
      \lineskip .5em%
      \begin{tabular}[t]{c}% <------
        \@author%            <------ Authors
      \end{tabular}\par}%    <------
    \vskip 1em%
  %  {\large \@date}%
  \end{center}%
  \par
  \vskip 1.5em}

\date{18/03/2019}

\title{\vspace{-2.2cm} \textbf{ELEN4020A: Data Intensive Computing \\ Laboratory Exercise 2}}
\author{\begin{tabular}{ll}
  Lynch Mwaniki & 1043475 \\
  Madimetja Sethosa & 1076467 \\
  Teboho Matsheke & 1157717 \\
\end{tabular}
 }


\begin{document}
%\centering

\maketitle
\thispagestyle{empty}\pagestyle{empty}
\vspace{-8mm}

\section{In-Place Matrix Transposition}
%
In this lab, three different In-place matrix transposition algorithms are implemented and tested against each other: naive approach, Diagonal and Block transpose algorithms. The Diagonal and Block transposition algorithms are implemented using Pthreads and the OpenMP library. The naive approach is implemented serially and using threading by implementing it using the OpenMP library.
%
\section{Basic Algorithm}
%
This algorithm uses two nested for loops to transpose a 2D matrix. The outer and inner loops are used to iterate through the rows and columns of the matrix respectively. They iterate along the main diagonal. Each value is swapped with its corresponding transposed location, A[i, j] with A[j, i]. This process only needs to be applied to the either the top or bottom triangular half of the matrix.\\
    ... place diagram
%
\section{Diagonal Transpose Algorithm}
%
This algorithm transposes the matrix elements by going along the major diagonal. As with the basic algorithm, each value is swapped with its corresponding transposed location, A[i, j] with A[j, i]. The openMP and pthread methods are implemented as follows: 
%
\subsection{Pthreads}
%
In implementing pthreads for the diagonal algorithm, we define a struct, ThreadDataDiagonal, with the following properties: diagonal Index, pointer to matrix, ID. We then define or set the number of threads to be used to 8 because of the .... In order to keep track of the row/column to be transposed, a global variable, nextDiagonal, is defined.\\

Each thread diagonal index is used to iterate through the rows and columns of the matrix while transposing the corresponding elements. Upon completion of this, a mutex lock section is created whereby, if the thread diagonal index is less than size of the matrix -1, it is set to nextDiagonal, thereafter nextDiagonal is incremented by one. Otherwise, the the thread diagonal index is set to size of the matrix -1 at which point the thread will be terminated. 


%
\subsection{OpenMP}
%
This implementation follows from the basic algorithm in that it uses two nested for loops to transpose a 2D matrix. The outer and inner loops are used to iterate through the rows via the main diagonal and columns of the matrix respectively. The two nested for loops are enclosed in a parallel section with their respective indexing variables made private. To ensure that the threads don't interchangeably transpose elements that have already been transposed and that they don't wait for others to finish executing, an auto schedule with a nowait instruction is instituted.
%
\section{Block Transpose Algorithm}
%
....
%
\subsection{Pthreads}
%
...
%
\subsection{OpenMP}
%
...
%
%
\section{Performance Comparison of different algorithms}
\begin{table}[H]
    \vspace{-0.5cm}
    \centering
    \caption{Results showing time taken to compute algorithms with 8 threads}
    \begin{tabular}{|l|l|l|l|l|l|l|}
    \hline
    \multicolumn{7}{|c|}{\textbf{No. of threads = 8}} \\ \hline
    \multirow{2}{*}{\textbf{$N_0$ = $N_1$}} & \multicolumn{1}{c|}{\textbf{Basic}} & \multicolumn{2}{c|}{\textbf{Pthreads}} & \multicolumn{3}{c|}{\textbf{OpenMP}} \\ \cline{2-7} 
     & \textbf{} & \textbf{Diagonal} & \textbf{Blocked} & \textbf{Naive} & \textbf{Diagonal} & \textbf{Blocked} \\ \hline
    \textbf{128} &  &  &  &  &  &  \\ \hline
    \textbf{1024} &  &  &  &  &  &  \\ \hline
    \textbf{2048} &  &  &  &  &  &  \\ \hline
    \textbf{4096} &  &  &  &  &  &  \\ \hline
    \end{tabular}
\end{table}
%
\begin{table}[H]
    \vspace{-0.5cm}
    \centering
    \caption{Results showing time taken to compute algorithms with 128 threads}
    \begin{tabular}{|l|l|l|l|l|l|l|}
    \hline
    \multicolumn{7}{|c|}{\textbf{No. of threads = 128}} \\ \hline
    \multirow{2}{*}{\textbf{$N_0$ = $N_1$}} & \multicolumn{1}{c|}{\textbf{Basic}} & \multicolumn{2}{c|}{\textbf{Pthreads}} & \multicolumn{3}{c|}{\textbf{OpenMP}} \\ \cline{2-7} 
     & \textbf{} & \textbf{Diagonal} & \textbf{Blocked} & \textbf{Naive} & \textbf{Diagonal} & \textbf{Blocked} \\ \hline
    \textbf{128} &  &  &  &  &  &  \\ \hline
    \textbf{1024} &  &  &  &  &  &  \\ \hline
    \textbf{2048} &  &  &  &  &  &  \\ \hline
    \textbf{4096} &  &  &  &  &  &  \\ \hline
    \end{tabular}
\end{table}
%
\section{Analysis of Results}
%
%
\section{Conclusion}
%
This report detailed three different matrix transposition algorithms implemented in the C programming language namely: Basic or naive approach, Diagonal and Block Transpose algorithms. The Diagonal and Block transposition algorithms were threaded and implemented using two different threading libraries namely: Pthreads and OpenMP. The Naive approach was also threaded using OpenMP. The algorithms were timed and the results were analyzed.
%
\clearpage
\appendix
\label{App:Apendix}
\section{Pseudo code}
\subsection{allocateMatrix}
    \begin{algorithm}[H]
        \caption{Allocate Matrix in Memory}
        \KwResult{Integer pointer, pointing to first element of matrix.}
        \SetKwInOut{Input}{Input}\SetKwInOut{Output}{Output}
        \Input{Total Number of Elements}
        result\_ptr = (int*)calloc(number\_of\_elements, sizeof(int)) \\
        \Return result\_ptr
    \end{algorithm}
%
\subsection{getElementLocation2D}
    \begin{algorithm}[H]
        \caption{Get Location of Element in 2D matrix when represented as 1D Array}
        \KwResult{Integer containing position of element in 1D matrix}
        \SetKwInOut{Input}{Input}\SetKwInOut{Output}{Output}
        \Input{1D Array containing indexes}
        \Input{Dimension N}
        row = index[0]\\
        column = index[1]\\
        \Return $(row*N)+column$
    \end{algorithm}
 % 
\subsection{retrieveElement2D}
    \begin{algorithm}[H]
        \caption{Retrieve Element from 2D matrix}
        \KwResult{Integer value of the element in the 2D matrix co-ordinates.}
        \SetKwInOut{Input}{Input}\SetKwInOut{Output}{Output}
        \Input{Pointer to matrix, 1D Array containing indexes}
        \Input{Dimension N}
        \Return $*(matrix\_ptr + getElementLocation2D(indexes, N))$
    \end{algorithm}
%
 % 
\subsection{blockElementsIndex}
    \begin{algorithm}[H]
        \caption{Retrieve block Elements indices}
        \KwResult{4 index values of a block in the matrix}
        \SetKwInOut{Input}{Input}\SetKwInOut{Output}{Output}
        \Input{blockIndex, Dimension N}
        Num\_Blocks = N/2\\
        *blockElemIndex =  (blockIndex*2)+ ((int)(blockIndex/Num\_Blocks))*N;\\
	    blockElemIndex[1]=*blockElemIndex+1; \\
	    blockElemIndex[2]=*blockElemIndex+N;\\
	    blockElemIndex[3]=*blockElemIndex+(N+1);\\
        \Return $blockElemIndex$
    \end{algorithm}
%
 % 
\subsection{swapBlocks}
    \begin{algorithm}[H]
        \caption{transpose two blocks}
        \KwResult{block transposed *matrix}
        \SetKwInOut{Input}{Input}\SetKwInOut{Output}{Output}
        \Input{*matrix, *blockA, *blockB}
        swapElements(matrix+indexBlockA[0], matrix+indexBlockB[0]);\\
	swapElements(matrix+indexBlockA[1], matrix+indexBlockB[1]);\\
	swapElements(matrix+indexBlockA[2], matrix+indexBlockB[2]);\\
	swapElements(matrix+indexBlockA[3], matrix+indexBlockB[3]);\\
    \end{algorithm}
%
\subsection{Basic Algorithm}
%
\begin{algorithm}[H]
    \caption{Transpose a square 2D Matrix using Naive Approach in Serial execution}
    \KwResult{A 2D square transposed matrix}
    \SetKwInOut{Input}{Input}\SetKwInOut{Output}{Output}
    \Input{Pointer to matrix}
    \Input{Matrix Size N}
    \For{ i = 0 to N - 1}
    {
        \For{ j = i to N-1}
        {
            temp = matrix element indexed at i and j \\
            matrix element at i and j position = matrix element indexed at j and i position\\
            matrix element indexed at j and i position = temp
        }
    }
\end{algorithm}
%
\subsection{OpenMP Naive Transpose Algorithm}
%
\begin{algorithm}[H]
    \caption{Transpose a square 2D Matrix using Naive Approach in Parallel execution}
    \KwResult{A 2D square transposed matrix}
    \SetKwInOut{Input}{Input}\SetKwInOut{Output}{Output}
    \Input{Pointer to matrix}
    \Input{Matrix Size N}
    set number of threads for OpenMP\\
    create variables i, j and temp\\
    \#pragma omp parallel shared(matrix) private(temp, i, j)\\
    \#pragma omp for schedule(dynamic) nowait\\
    \For{ i = 0 to N - 1}
    {
        \For{ j = i to N-1}
        {
            temp = matrix element indexed at i and j \\
            matrix element at i and j position = matrix element indexed at j and i position \\
            matrix element indexed at j and i position = temp
        }
    }
\end{algorithm}
%
\subsection{OpenMP Diagonal Transpose Algorithm}
%
\begin{algorithm}[H]
    \caption{Transpose a square 2D Matrix using Naive Approach in Parallel execution}
    \KwResult{A 2D square transposed matrix}
    \SetKwInOut{Input}{Input}\SetKwInOut{Output}{Output}
    \Input{Pointer to matrix}
    \Input{Matrix Size N}
    set number of threads for OpenMP\\
    create variables diagonalIndex, j and temp\\
    \#pragma omp parallel shared(matrix) private(temp, i, j)\\
    \#pragma omp for schedule(dynamic) nowait\\
    \For{ diagonalIndex = 0 to N - 2}
    {
        \For{ j = diagonalIndex + 1 to N-1}
        {
            temp = matrix element indexed at diagonalIndex and j \\
            matrix element at diagonalIndex and j position = matrix element indexed at j and diagonalIndex position \\
            matrix element indexed at j and diagonalIndex position = temp
        }
    }
\end{algorithm}
%
\subsection{OpenMP Block Transpose Algorithm}
%

%
\subsection{Pthread Diagonal Transpose Algorithm}
%
\begin{algorithm}[H]
    \caption{Transpose a square 2D Matrix using Diagonal Algorithm}
    \KwResult{A 2D square transposed matrix}
    \SetKwInOut{Input}{Input}\SetKwInOut{Output}{Output}
    set number of threads for Pthreads\\
    create global variable \textbf{nextDiagonal} and set it to number of threads \\
    struct ThreadData{ int* matrix; int diagIndex;} \\
    
    \textbf{in main:} \\
    Allocate memory for 2D matrix of size N and fill with random integers \\
    Create 1D matrix of size number of Pthreads for storing structs of type ThreadData \\
    Create pthread threads with number assigned above.\\
    
    \ForEach{thread}
    {   
        Assign threadData's matrix pointer to the 2D matrix created earlier.\\
        Assign threadData's diagIndex to the number of the thread being created.\\
        \While{true}
        {
            \For{ j = diagIndex + 1 to N - 1}
            {
                temp = matrix element indexed at diagIndex and j \\
                get blockA element indices\\
                transpose elements within blockA\\
                \If{j!=diagIndex}{
                    temp = matrix element indexed at j and diagIndex\\
                    get blockB element indices\\
                    transpose elements within blockB\\
                    transpose blockA and blockB
                }
            }
            
            Mutex Lock\\
            %
            \If{nextDiagonal == N - 1} {diagIndex = N - 1}
            \Else
            { 
                diagnIndex = nextDiagonal \\
                nextDiagonal = nextDiagonal + 1
            }
            %
            Mutex Unlock\\
            
            \If{diagIndex == N - 1}{ break from while loop}
        }
        exit pthread
    }
    
    \ForEach{thread}
    {
        Join thread and wait for other threads to exit and join
    }
\end{algorithm}
%
\subsection{Pthread Block Transpose Algorithm}
%
\begin{algorithm}[H]
    \caption{Transpose a square 2D Matrix using Block Transpose Algorithm}
    \KwResult{A 2D square transposed matrix}
    \SetKwInOut{Input}{Input}\SetKwInOut{Output}{Output}
    set number of threads for Pthreads\\
    create global variable \textbf{nextDiagBlock} \\
    struct ThreadData{ int* matrix; int diagIndex;} \\
    
    \textbf{in main:} \\
    Allocate memory for 2D matrix of size N and fill with random integers \\
    Create 1D matrix of the size of number Pthreads for storing structs of type ThreadData \\
    Create pthread threads with number assigned above.\\
    Calculate the number of blocks (N\_blocks)\\
    \ForEach{thread}
    {   
        Assign threadData's matrix pointer to the 2D matrix created earlier.\\
        Assign threadData's diagIndex to the number of the thread being created.\\
        \While{true}
        {
            \For{ j = diagIndex to N\_blocks}
            {
                temp = matrix element indexed at diagIndex and j \\
                get blockA element indices using the blockElementsIndex function\\
                transpose the elements in blockA\\
                \If{j!=diagIndex}{
                    temp = matrix element indexed at j and diagIndex\\
                    get blockB element indices using the blockElementsIndex function\\
                    transpose the elements in blockB\\
                    
                    transpose blockA and blockB using the swapBlocks() function.
                }
            }
            
            Mutex Lock\\
            %
            \If{nextDiagBlock $<$ N\_blocks - 1} {diagIndex =nextDiagBlock++;}
            \Else{diagIndex=N\_blocks - 1}
            %
            Mutex Unlock\\
            
            \If{diagIndex == N\_blocks - 1}{ break from while loop}
        }
        exit pthread
    }
    
    \ForEach{thread}
    {
        Join thread and wait for other threads to exit and join
    }
\end{algorithm}
%
\end{document}